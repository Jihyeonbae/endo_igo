

\documentclass[12pt]{article}
\usepackage[margin=1in]{geometry}
\usepackage{newtxtext,newtxmath} % Better than \usepackage{times}
\usepackage{textgreek}
\usepackage{hyperref}
\usepackage{setspace}
\doublespacing
\usepackage{booktabs}
\usepackage{threeparttable}
\usepackage[table]{xcolor}
\usepackage{graphicx}
\usepackage{appendix}
\usepackage{siunitx}
\usepackage{tabularray}
\usepackage{indentfirst}

\sisetup{detect-all} % ensures consistency with font settings
\usepackage[style=apa, backend=biber]{biblatex}
\addbibresource{references.bib}



\title{Who Changes IGOs? Endogenous Drift and Entry Shocks in Organizational Regime Orientation}

\author{Jihyeon Bae}

\date{\today\\[0.5em] }

\begin{document}

\maketitle

\begin{abstract}
When IGOs become more authoritarian (or democratic), is the change driven by many small movements among incumbents or by few large compositional shocks? Using new distributional measures of IGO regime orientation from 1960–2019, I show that regime-crossing is common and that endogenous changes among incumbent members account for a substantial share of organizational transformation, even when membership turnover is rare. These findings challenge theories that treat regime-based IGOs as static institutional types.

\end{abstract}

\section*{Research Question}
Inter-governmental Organizations (IGOs) that are composed of democracies boost inchoate democracies' stability \parencite[]{pevehouse_democracy_2005}. Likewise, IGOs composed of autocracies are supposed to help autocrats sustain longer \parencite[]{debre_clubs_2022}. If so, we should rarely observe IGOs' dominant power shift, i.e., autocracy- to democracy-skewed IGOs. Existing theories implicitly treat regime-based IGOs as stable institutional equilibria, in which organizational orientation is largely determined at founding or through membership turnover. I tackle this empirical question by reviewing literature on regime-based IGO theories and offer explanations on endogenous changes of IGOs.

While it is mechanically true that an IGO’s aggregate democracy level reflects its members, existing work rarely examines how such changes occur or which processes dominate organizational transformation. IGO's ``regime-crossing" merits further investigation for three reasons. First, policy implications for approaching DIGOs and AIGOs differ significantly, and understanding when and whether IGOs ever cross these categories can offer useful implications for policy space. Second, understanding whether AIGOs' slide into DIGOs (and less frequent, but vice versa) can be a hard test for the AIGOs' efficacy. Third, distinguishing endogenous drift from exogenous shocks speaks directly to theories of authoritarian cooperation: if AIGOs are resilient to incumbent change but sensitive to entry, they function differently than if gradual internal change drives transformation.

\subsection*{Endogenous}
For one, IGOs' membership composition shift when members' domestic regime types change. 

\subsection*{Exogenous}
Second, IGOs can cross from autocracy to democracy type when new members join. 

\subsection*{Data and Methods}
I use original measurement of IGOs' distributional shapes; autocracy-skewed, democracy-skewed, unimodal, and polarized (U-shaped). Elsewhere, I introduce this measurement of IGO's distributional characteristics based on its membership component. With this, I first show how often IGOs' regime-crossing actually happens. 

\begin{figure}[ht]
  \centering
  \includegraphics[width=\textwidth]{overleaf/aigo_trajectory.png}
  \caption{AIGO's change in democracy score over time. 
  Points show year-to-year estimated IGO democracy mean ($ \mu$).}
  \label{fig:aigo_trajectory}
\end{figure}

\begin{figure}[ht]
  \centering
  \includegraphics[width=\textwidth]{overleaf.png}
  \caption{DIGO's change in democracy score over time. 
  Points show year-to-year estimated IGO democracy mean ($ \mu$).}
  \label{fig:digo_trajectory}
\end{figure}

\begin{figure}[ht]
  \centering
  \includegraphics[width=\textwidth]{overleaf/igo_regime_orientation_changes.png}
  \caption{IGO regime orientation changes persist over time. 
  Points show year-to-year changes in the estimated IGO democracy mean ($\Delta \mu$). 
  Colored points indicate regime-orientation crossings.}
  \label{fig:igo_regime_changes}
\end{figure}


As \ref{fig:igo_regime_changes} shows, IGOs have shown changes of their gravity center. What remains unanswered is whether this change is driven by new members or existing members' democratization. 


\subsection*{Results}
\vspace{1em}


\end{document}
